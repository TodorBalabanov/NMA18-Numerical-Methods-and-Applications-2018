\documentclass{llncs}

\usepackage{graphicx}
\usepackage{placeins}

\begin{document}

\title{Optimization of String Rewriting Operations for 3D Fractal Generation with Genetic Algorithms}

\author{Todor Balabanov, Janeta Sevova, Kolyu Kolev}

% Todor Balabanov todorb@iinf.bas.bg
% Kolyu Kolev kolev_kolyu@yahoo.com
% Janeta Sevova janetasevova@gmail.com

\institute{Institute of Information and Communication Technologies \\
Bulgarian Academy of Sciences \\
acad. Georgi Bonchev Str., block 2, office 514, 1113 Sofia, Bulgaria \\
\email{todorb@iinf.bas.bg} \\
\texttt{http://www.iict.bas.bg/}}

%----------------------------------------------------------------------------------------
%   Title
%----------------------------------------------------------------------------------------
\maketitle

%----------------------------------------------------------------------------------------
%   Abstract
%----------------------------------------------------------------------------------------
\begin{abstract}
String rewriting is a modification of the idea for the context-free grammar. Modification consists in the fact that there is not separation on terminal and nonterminal symbols. Each symbol in string rewriting is considered as nonterminal and it can produce longer sequence. By this way infinite structures are created as fractals. In a video presentation Jack Hodkinson suggested the idea instead of text symbols pixels in 2D image to be used. Each pixel (geometric square) is divided in nine sub-squares. The color of the pixel determine the pattern in which the nine squares are arranged. By this way each pixel gives a rule for subdivision of the area under it. The goal of Hodkinson's investigation is how to setup the rules in such way that when the rules are applied fractal to reproduce a particular 2D image (in particular case the glyph of the Greek letter $\pi$). Because it is an optimization problem a good option was a genetic algorithm and this is exactly what Hodkinson did. The genetic algorithm was used to assemble set of rules for string rewriting with 2D pixels. Hodkinson did series of experiments with the size of substitution matrix (not only 3x3, but also 2x2 and 4x4). He also did series of experiments with different number of colors, starting with black/white and reaching the final successful experiment with thirty shades of blue. In this study the idea for string rewriting is extended in the 3D space and instead of pixels voxels are used with 3x3x3 substitution matrix. The study is only on virtual models but with the usage of industrial tomography 3D objects can be scanned and with 3D color printer generated fractals can be printed.

\keywords{genetic algorithms, fractals generation, string rewriting}
\end{abstract}

%----------------------------------------------------------------------------------------
%   Paper
%----------------------------------------------------------------------------------------
\section{Introduction} \label{Introduction}

There is a branch in the theory for the formal languages which is related to a context-free grammar (CFG). In such grammar there are set of production rules. Application of these rules lead to generation of all possible strings in a specified formal language. In CFG a production rule is an operation of simple replacement. Rules are applied regardless of the context. The left hand side of a CFG rule is always nonterminal symbol. It means that nonterminal symbols are not part of the resulting string. The right hand side of a CFG rule consists of terminal symbol, nonterminal symbols or combination of both. The main idea in CFGs is that nonterminal symbols are substituted with the given rules until there is no nonterminal symbols in the generated string. 

String rewriting system (SRS) is a substitution system in which rules does not contain nonterminal symbols. Each terminal symbol can appear on the LHS of the rule and on the RHS of the rule. With such definition SRS are useful for generation of infinite structures and they are mainly used for the generation of some fractals (for example: Box Fractal, Cantor Dust, Cantor Square Fractal and Sierpinski Carpet). Most of the fractals are generated with binary rules (black and white colors), but in the Hodkinson's research it is clearly shown that limit of the colors can be higher. 

\begin{figure}[h]
  \centering
  \includegraphics[width=1.0\linewidth]{pic01}
  \caption{Fractal generated by Jack Hodkinson.}
\label{fig:pic01}
\end{figure}

In Hodkinson's research genetic algorithms (GAs) were employed in order 2D shape of Greek letter $\pi$ to be reconstructed with substitution rules of 30 shades of blue pixels (Fig. \ref{fig:pic01}). The key problem in his research was wich 30 rules to be selected in order the final goal to be achieved. GAs were a promising approach for such kind of combinatorial optimization. A single rule had LHS pixel with particular color and nine pixels on the RHS as 3x3 substitution matrix. Image starts with a single pixel and it grows on size by 3 on each iteration. 

This study extends the same idea from 2D space into 3D space. Voxels are used instead of pixels. Square matrix of 3x3 is replaced with cube 3x3x3. The paper is organized as follows: Section \ref{Introduction} introduces the problem; Section \ref{Model and Optimization} presents a model and optimization approach; Section \ref{Experiments and Results} gives some experiment details and some results are shown; Section \ref{Conclusions} concludes and some further ideas for research are pointed.

\section{Model and Optimization} \label{Model and Optimization}

\section{Experiments and Results} \label{Experiments and Results}

\section{Conclusions} \label{Conclusions}

%----------------------------------------------------------------------------------------
%   Acknowledgements
%----------------------------------------------------------------------------------------
\section*{Acknowledgements}
This work was supported by private funding of Velbazhd Software LLC.

%----------------------------------------------------------------------------------------
%   Bibliography
%----------------------------------------------------------------------------------------
\begin{thebibliography}{99}
\end{thebibliography}

\end{document}